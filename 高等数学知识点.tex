\documentclass[a4paper,12pt,UTF8]{ctexart}
\usepackage{MyStyle}

\begin{document}
    \title{高等数学知识点}
    \author{王泠风}
    \date{2024 年 10 月 04 日}
    \maketitle

    \section{基础公式}
    
    \subsection{不等式}
    \begin{align*}
        \left| a\pm b \right|& \leq \left| a \right| + \left| b \right|& \left| \left| a \right| - \left| b \right| \right|& \leq \left| a-b \right|& \sqrt{ab}& \leq \frac{a+b}{2} \leq \sqrt{\frac{a^2+b^2}{2}}&\\
        \sin x& \leq x \leq \tan x& \frac{1}{1+x}& \leq \ln \left( 1+\frac{1}{x} \right) \leq \frac{1}{x}& \ln \left( 1+x \right)& \leq x \leq e^x-1&\\
    \end{align*}

    \subsection{三角函数}
    \begin{align*}
        \tan \frac{x}{2}& = \pm \sqrt{\frac{1-\cos x}{1+\cos x}} = \frac{1-\cos x}{\sin x}& && \tan \frac{x}{2}& = u&\\
        \sin \frac{x}{2}& = \pm \sqrt{\frac{1-\cos x}{2}}& \tan (\alpha + \beta)& = \frac{\tan \alpha + \tan \beta}{1 - \tan \alpha \tan \beta}& \sin x& = \frac{2u}{1+u^{2}}&\\
        \cos \frac{x}{2}& = \pm \sqrt{\frac{1+\cos x}{2}}& \tan (\alpha - \beta)& = \frac{\tan \alpha - \tan \beta}{1 + \tan \alpha \tan \beta}& \cos x& = \frac{1-u^{2}}{1+u^{2}}&\\
    \end{align*}

    \section{极限}

    \subsection{两个重要极限}
    \begin{align*}
        \lim_{x\to 0}\frac{\sin x}{x}& = 1& \lim_{x\to\infty}\left(1+\frac{1}{x}\right)^x& = \mathrm{e}&\\
    \end{align*}
    
    \subsection{泰勒公式极限应用(x→0)}
    \begin{align*}
        \sin x& = x - \frac{x^{3}}{6} + o(x^{3})& \arcsin x& = x + \frac{x^{3}}{6} + o(x^{3})&\\
        \tan x& = x + \frac{x^{3}}{3} + o(x^{3})& \arctan x& = x - \frac{x^{3}}{3} + o(x^{3})&\\
        \cos x& = 1 - \frac{x^{2}}{2} + \frac{x^{4}}{24} + o(x^{4})& \ln (1+x)& = x - \frac{x^{2}}{2} + \frac{x^{3}}{3} + o(x^{3})&\\
        e^x& = 1 + x + \frac{x^{2}}{2!} + \frac{x^{3}}{3!} + o(x^{3})& (1+x)^{a}& = 1 + ax + \frac{a(a-1)}{2!}x^{2} + o(x^{2})&\\
    \end{align*}

    \subsection{极限不存在}
    极限趋于无穷、只存在一个极限、左右极限不等、震荡不存在、函数无定义    

    \subsection{求极限}

    \paragraph{极限转化为积分(2017年16题)} \(\lim \limits_{x \to \infty} \sum \limits_{k=1}^{n}\frac{k}{n^2}\ln (1+ \frac{k}{n}) \)  极限可能是积分的极限形式,转化为积分处理。

    \section{一元函数微分学}

    \subsection{反函数导数}
    \begin{align*}
        x_y'& = \frac{1}{y_x'}& x_{yy}''& = \frac{-y_{xx}''}{(y_x')^{3}}&\\
    \end{align*}

    \subsection{常用求导公式}
    \begin{align*}
        (\sec x)^{'}& = \sec x\tan x& (\arcsin x)^{'}& = \frac{1}{\sqrt{1-x^{2}}}&\\
        (\csc x)^{'}& = -\csc x\cot x& (\arccos x)^{'}& = -\frac{1}{\sqrt{1-x^{2}}}&\\
        (\tan x)^{'}& = \sec^{2}x& (\arctan x)^{'}& = \frac{1}{1+x^{2}}&\\
        (\cot x)^{'}& = -\csc^{2}x& (\operatorname{arccot} x)^{'}& = -\frac{1}{1+x^{2}}&\\
        [\ln(x+\sqrt{x^{2}+1})]^{'}& = \frac{1}{\sqrt{x^{2}+1}}& [\ln(x+\sqrt{x^{2}-1})]^{'}& = \frac{1}{\sqrt{x^{2}-1}}&\\
    \end{align*}

    \subsection{中值定理}
    \paragraph{涉及函数的中值定理}:有界与最值定理、介值定理、平均值定理、零点定理
    \paragraph{涉及导数的中值定理}:费马定理、罗尔定理、拉格朗日中值定理、柯西中值定理、泰勒公式。

    \subsubsection{罗尔定理}
    \begin{align*}
        f'(x)+kf(x)&\Rightarrow f(x)e^{kx}& [f^2(x)]'& = 2f(x)f'(x)& [f(x)f'(x)]'& = [f'(x)]^2+f(x)f''(x)&\\
    \end{align*}

    \subsubsection{泰勒公式}

    \paragraph{泰勒原式}
    \begin{align*}
        f(x)& = f(x_0)+f'(x_0)(x-x_0)+\frac{f''(x_0)}{2!}(x-x_0)^2+\cdots+\frac{f^{(n)}(x_0)}{n!}(x-x_0)^n+\frac{f^{(n+1)}(\xi)}{(n+1)!}(x-x_0)^{n+1}&\\
    \end{align*}

    \paragraph{泰勒展开式}
    \begin{align*}
        e^x& = \sum_{n=0}^{\infty}\frac{x^{n}}{n!}&& = 1 + x + \frac{x^{2}}{2!} + \frac{x^{3}}{3!} + \cdots + \frac{x^{n}}{n!} + o(x^{n}) \enspace (-\infty < x < +\infty)&\\
        \sin x& = \sum_{n=0}^{\infty}(-1)^{n}\frac{x^{2n+1}}{(2n+1)!}&& = x - \frac{x^{3}}{3!} + \frac{x^{5}}{5!} - \cdots + (-1)^{n}\frac{x^{2n+1}}{(2n+1)!} + o(x^{2n+1}) \enspace (-\infty < x < +\infty)&\\
        \cos x& = \sum_{n=0}^{\infty}(-1)^{n}\frac{x^{2n}}{(2n)!}&& = 1 - \frac{x^{2}}{2!} + \frac{x^{4}}{4!} - \cdots + (-1)^{n}\frac{x^{2n}}{(2n)!} + o(x^{2n}) \enspace (-\infty < x < +\infty)&\\
        \frac{1}{1-x}& = \sum_{n=0}^{\infty}x^{n}&& = 1 + x + x^{2} + x^{3} + \cdots + x^{n} + o(x^{n}) \enspace (-1 < x < 1)&\\
        \frac{1}{1+x}& = \sum_{n=0}^{\infty}(-1)^{n}x^{n}&& = 1 - x + x^{2} - x^{3} + \cdots + (-1)^{n}x^{n} + o(x^{n}) \enspace (-1 < x < 1) &\\
        \ln (1+x)& = \sum_{n=1}^{\infty}(-1)^{n-1}\frac{x^{n}}{n}&& = x - \frac{x^{2}}{2} + \frac{x^{3}}{3} - \cdots + (-1)^{n-1}\frac{x^{n}}{n} + o(x^{n}) \enspace (-1 < x \leq 1)&\\
    \end{align*}
    \begin{align*}
        (1+x)^{a}& = 1 + ax + \frac{a(a-1)}{2!}x^{2} + \cdots + \frac{a(a-1)\cdots(a-n+1)}{n!}x^{n} + o(x^{n}) \enspace 
        \begin{cases}
            x\in (-1,1),&\text{当}a\leq -1,\\
            x\in (-1,1],&\text{当}-1<a<0,\\
            x\in [-1,1],&\text{当}a>0\text{且}a\notin \mathbb{N_+},\\
            x\in \mathbb{R},&\text{当}a\in \mathbb{N_+}.
        \end{cases}&\\
    \end{align*}

    \subsection{求导数}

    \subsubsection{求高阶导数}

    \paragraph{归纳法}

    \paragraph{高阶求导公式}

    \paragraph{泰勒公式(2017年9题)}
    将 \(f(x)=\frac{1}{1+x^2}\) 化为已知的泰勒展开式,再通过比较系数求出 \(f^{(n)}(x_0)\)。

    \section{一元函数积分学}

    \subsection{常用积分公式}
    \begin{align*}
        \int {a}^{x} dx& = \frac{{a}^{x}}{\ln a}+C& \int \frac{1}{a^{2}+x^{2}}dx& = \frac{1}{a}\arctan\frac{x}{a}+C \enspace (a>0)&\\
        \int \tan x dx& = -\ln \left|\cos x \right|+C& \int \frac{1}{\sqrt{a^{2}-x^{2}}}dx& = \arcsin\frac{x}{a}+C \enspace (a>0)&\\
        \int \cot x dx& = \ln \left|\sin x \right|+C& \int \frac{1}{\sqrt{x^{2}+a^{2}}}dx& = \ln\left(x+\sqrt{x^{2}+a^{2}}\right)+C&\\
        \int \sec x dx& = \ln \left|\sec x + \tan x \right|+C& \int \frac{1}{\sqrt{x^{2}-a^{2}}}dx& = \ln\left|x+\sqrt{x^{2}-a^{2}}\right|+C \enspace (\left|x\right|>\left|a\right|)&\\
        \int \csc x dx& = \ln \left|\csc x - \cot x \right|+C& \int \frac{1}{x^{2}-a^{2}}dx& = \frac{1}{2a}\ln\left|\frac{x-a}{x+a}\right|+C&\\
        \int \tan^2 x dx& = \tan x-x+C& \int \cot^2 x dx& = -\cot x-x+C& \\
    \end{align*}
    
    \subsection{点火公式}
    \begin{align*}
        \int_{0}^{\frac{\pi}{2}} \sin^{n}x dx = \int_{0}^{\frac{\pi}{2}} \cos^{n}x dx =
        \begin{cases}
            \frac{n-1}{n} \times \frac{n-3}{n-2} \times \cdots \times \frac{2}{3} \times 1,& n \text{为大于} 1 \text{的奇数},\\
            \frac{n-1}{n} \times \frac{n-3}{n-2} \times \cdots \times \frac{1}{2} \times \frac{\pi}{2},& n \text{为正偶数}.
        \end{cases}
    \end{align*}
    积分0到\(\pi\)和0到\(2\pi\)根据被积函数在被积区间的正负性判断是否为0。

    \section{多元函数微分学}
    
    \subsection{链式求导法则}

    \subsection{隐函数求导法则}

    \paragraph{隐函数求导法则1}当 \(F(x,y)=0\) 时 \(\frac{dy}{dx}=-\frac{F'_x}{F'_y}\)

    \paragraph{隐函数求导法则2}当 \(F(x,y,z)=0\) 时 \(\frac{\partial z}{\partial x}=-\frac{F'_x}{F'_z}\) \(\frac{\partial z}{\partial y}=-\frac{F'_y}{F'_z}\)

    \subsection{二元函数的极值}

    \subsubsection{无条件极值}

    \paragraph{二元函数取极值的充分条件}
    记 
        \(\begin{cases}
            f''_{xx}(x_0,y_0) = A,\\
            f''_{xy}(x_0,y_0) = B,\\
            f''_{yy}(x_0,y_0) = C,\\
        \end{cases}\)
    则 \(\Delta = AC-B^2\) 
        \(\begin{cases}
            >0 \Rightarrow \text{极值}
            \begin{cases}
                A<0 \Rightarrow &\text{极大值}\\
                A>0 \Rightarrow &\text{极小值}\\
            \end{cases}\\
            <0 \Rightarrow \text{非极值}\\
            =0 \Rightarrow \text{方法失效}\\
        \end{cases}\)
    
    方法失效时通常不是极值,可利用极值定义通过举反例证明,常用反例为 \(x=y\)、\(x=-y\)和\(y=0\)。

    \subsubsection{条件极值}
    求目标函数 \(u=f(x,y,z)\) 在条件 
        \(\begin{cases}
            \varphi (x,y,z)=0,\\
            \psi (x,y,z)=0\\
        \end{cases}\) 
    下的极值。
        
    构造辅助函数 \(F(x,y,z,\lambda,\mu)=f(x,y,z)+\lambda \varphi (x,y,z)+\mu \psi (x,y,z)\)

    令
    \(\begin{cases}
        F'_x=f'_x+\lambda \varphi '_x + \mu \psi '_x=0,\\
        F'_y=f'_y+\lambda \varphi '_y + \mu \psi '_y=0,\\
        F'_z=f'_z+\lambda \varphi '_z + \mu \psi '_z=0,\\
        F'_\lambda=\varphi (x,y,z)=0,\\
        F'_\mu=\psi (x,y,z)=0;\\
    \end{cases} \)

    解得 \(x,y,z,\lambda,\mu\) 的值,代入 \(f(x,y,z)\) 即可。
     
    \section{多元函数积分学}

    \subsection{普通对称性和轮换对称性}
    普通对称性需要观察函数特征,轮换对称性一定成立,但只有在区域D轮换对称后不变的情况下才有化简的作用。

    \subsection{坐标系转换}
    直角坐标系、极坐标系、球坐标系

    \begin{align*}
        \text{令}&
        \begin{cases}
            x=r \cos \theta\\
            y=r \sin \theta\\
        \end{cases}&
        \iint \limits_{\Sigma } f(x,y)dxdy& = \iint \limits_{\Sigma } f(r \cos \theta,r \sin \theta) rdrd\theta&\\
        \text{令}&
        \begin{cases}
            x=r \sin \varphi \cos \theta \\
            y=r \sin \varphi \sin \theta \\
            z=r \cos \varphi \\
        \end{cases}&
        \iiint \limits_{\Omega} f(x,y,z)dxdydz& = \iiint \limits_{\Omega} f(r \sin \varphi \cos \theta,r \sin \varphi \sin \theta,r \cos \varphi )r^2 \sin \varphi dr d\varphi d\theta&
    \end{align*}

    \section{常微分方程}
    
    \subsection{一阶微分方程}

    \subsubsection{变量可分离型和可化为变量可分离型}

    \subsubsection{一阶线性微分方程}
    \paragraph{方程结构}\(y'+p(x)y=q(x)\)
    \paragraph{通解公式}
    \[y=e^{-\int p(x)dx}\left[\int e^{\int p(x)dx} \cdot q(x)dx+C\right]\]

    \subsubsection{伯努利方程}
    方程结构为 \(y'+p(x)y=q(x)y^{n}\),令 \(z=y^{1-n}\) 可推导出的结构为 \(z'+(1-n)p(x)z=(1-n)q(x)\) 即一个一阶线性微分方程,求z然后代换为y。
    
    \subsection{二阶可降阶微分方程}
    
    \subsubsection{y''=f(x,y')型(方程中不显含未知函数y)}
    令 \(y'=p(x)\) , \(y''=p'\) ,求p然后积分 
    
    \subsubsection{y''=f(y,y')型(方程中不显含自变量x)}
    令 \(y'=p\) , \(y''=\frac{dp}{dy} \cdot p\) ,求p然后积分  
    
    \subsection{高阶线性微分方程}
    
    \subsubsection{二阶常系数齐次线性微分方程}
    \paragraph{方程结构} \(y''+py'+qy=0\)

    \(r^2+pr+q=0\) 求 \(\Delta=p^2-4q\)
    \paragraph{\(\Delta>0\)} \(y=C_1e^{r_1x}+C_2e^{r_2x}\)
    \paragraph{\(\Delta=0\)} \(y=(C_1+C_2x)e^{rx}\)
    \paragraph{\(\Delta<0\)} \(y=e^{\alpha x}(C_1\cos\beta x+C_2\sin\beta x)\)
    
    \subsubsection{二阶常系数非齐次线性微分方程}
    \paragraph{方程结构} \(y''+py'+qy=f(x)\)

    当 \(f(x)=P_n(x)e^{\alpha x}\) ,设 \(y=e^{\alpha x}Q_n(x)x^k\) ,其中
    \(\begin{cases}
        e^{\alpha x}\text{照抄},\\
        Q_n(x)\text{为x的n次多项式},\\
        k=
        \begin{cases}
            0, &\alpha\text{不是特征根}\\
            1, &\alpha\text{是单特征根}\\
            2, &\alpha\text{是二重特征根}\\
        \end{cases} 
    \end{cases}\)
    
    当 \(f(x)=e^{\alpha x}\left[P_m(x)\cos \beta x + P_n(x)\sin \beta x\right]\) ,设 \(y=e^{\alpha x} \left[Q_l^{(1)}(x)\cos \beta x + Q_l^{(2)}(x)\sin \beta x\right]x^k\) ,
    
    其中
    \(\begin{cases}
        e^{\alpha x}\text{照抄},\\
        l=max\{m,n\},Q_l^{(1)}(x),Q_l^{(2)}(x)\text{分别为x的两个不同的l次多项式},\\
        k=
        \begin{cases}
            0, &\alpha\pm\beta i \text{不是特征根},\\
            1, &\alpha\pm\beta i \text{是特征根}.\\
        \end{cases} 
    \end{cases}\)

    \subsubsection{n阶常系数齐次线性微分方程}

    \subsubsection{欧拉方程}
    \paragraph{方程结构} \(x^{2}\frac{d^2y}{dx^2}+px\frac{dy}{dx}+qy=f(x)\)
    \paragraph{固定解法}
    \subparagraph{当 \(x>0\) 时} 令 \(x=e^t\) 方程化为 \[\frac{d^2y}{dt^2}+(p-1)\frac{dy}{dt}+qy=f(e^t)\] 即二阶常系数非齐次线性微分方程
    \subparagraph{当 \(x<0\) 时} 令 \(x=-e^t\) 方程化为 \[\frac{d^2y}{dt^2}+(p-1)\frac{dy}{dt}+qy=f(-e^t)\] 即二阶常系数非齐次线性微分方程

    \section{无穷级数}

    \subsection{级数敛散性判别}

    \subsubsection{正项级数}
    收敛原则、比较判别法、比较判别法极限形式、比值判别法、根值判别法

    \subsubsection{交错级数}

    \paragraph{莱布尼茨判别法} 交错级数 \(\{u_n\}\) 单调不增且 \(\lim \limits_{x \to \infty} u_n=0\) ,则该级数收敛。

    \subsubsection{任意项级数}

    \begin{Definition}
        设 \(\sum \limits_{n=1}^{\infty} u_n\) 为任意项级数,若 \(\sum \limits_{n=1}^{\infty} |u_n|\) 收敛,则称 \(\sum \limits_{n=1}^{\infty} u_n\) 绝对收敛。
    \end{Definition}
    \begin{Definition}
        设 \(\sum \limits_{n=1}^{\infty} u_n\) 为任意项级数,若 \(\sum \limits_{n=1}^{\infty} u_n\) 收敛, \(\sum \limits_{n=1}^{\infty} |u_n|\) 发散,则称 \(\sum \limits_{n=1}^{\infty} u_n\) 条件收敛。
    \end{Definition}
    \begin{Theorem}
        若任意项级数 \(\sum \limits_{n=1}^{\infty} u_n\) 绝对收敛,则 \(\sum \limits_{n=1}^{\infty} u_n\) 必收敛。
    \end{Theorem}    

    \subsubsection{常用抽象数项级数敛散性}
    \begin{itemize}
        \item 设 \(\sum \limits_{n=1}^{\infty } \left| u_n \right| \) 收敛,则 \(\sum \limits_{n=1}^{\infty } u_n\) 收敛;设 \(\sum \limits_{n=1}^{\infty } u_n\) 发散,则 \(\sum \limits_{n=1}^{\infty } \left| u_n \right| \) 发散。
        \item 设 \(\sum \limits_{n=1}^{\infty } u_n^2\) 收敛,则 \(\sum \limits_{n=1}^{\infty } \frac{u_n}{n}\) 绝对收敛。
        \item 设 \(\sum \limits_{n=1}^{\infty } u_n\) 收敛,则 \(\sum \limits_{n=1}^{\infty } \left| u_n \right| \) 不定。
        \item 设 \(\sum \limits_{n=1}^{\infty } u_n\) 收敛,则 \(\sum \limits_{n=1}^{\infty } u_n^2 \) 不定。
        \item 设 \(\sum \limits_{n=1}^{\infty } u_n\) 收敛,则 \(\sum \limits_{n=1}^{\infty } (-1)^{n} u_n \) 不定。
        \item 设 \(\sum \limits_{n=1}^{\infty } u_n\) 收敛,则 \(\sum \limits_{n=1}^{\infty } (-1)^{n} \frac{u_n}{n} \) 不定。
        \item 设 \(\sum \limits_{n=1}^{\infty } u_n\) 收敛,则 \(\sum \limits_{n=1}^{\infty } u_{2n} \) , \(\sum \limits_{n=1}^{\infty } u_{2n-1} \) 不定。
        \item 设 \(\sum \limits_{n=1}^{\infty } u_n\) 收敛,则 \(\sum \limits_{n=1}^{\infty } (u_{2n-1} + u_{2n}) \) 收敛。
        \item 设 \(\sum \limits_{n=1}^{\infty } u_n\) 收敛,则 \(\sum \limits_{n=1}^{\infty } (u_{2n-1} - u_{2n}) \) 不定。
        \item 设 \(\sum \limits_{n=1}^{\infty } u_n\) 收敛,则 \(\begin{cases}
            \sum \limits_{n=1}^{\infty } (u_{n} + u_{n+1}) \text{收敛}, \sum \limits_{n=1}^{\infty } u_{n} + \sum \limits_{n=1}^{\infty } u_{n+1} \text{收敛},\\
            \sum \limits_{n=1}^{\infty } (u_{n} - u_{n+1}) \text{收敛}, \sum \limits_{n=1}^{\infty } u_{n} - \sum \limits_{n=1}^{\infty } u_{n+1} \text{收敛}.\\
        \end{cases} \) 
        \item 设 \(\sum \limits_{n=1}^{\infty } u_n\) 收敛,则 \(\sum \limits_{n=1}^{\infty } u_{n}u_{n+1} \) 不定。
    \end{itemize}

    \subsubsection{常用级数的敛散性}
    \begin{align*}
        \text{P级数}\enspace& \sum_{n=1}^{\infty}\frac{1}{n^{p}}
        \begin{cases}
            p>1,&\text{收敛},\\
            p \leq 1,&\text{发散}.
        \end{cases}&
        \text{P积分}\enspace& \int_{1}^{+\infty}\frac{1}{x^{p}}dx
        \begin{cases}
            p>1,&\text{收敛},\\
            p \leq 1,&\text{发散}.
        \end{cases}&\\
        \text{广义P级数}\enspace& \sum_{n=2}^{\infty}\frac{1}{n \ln^p n}
        \begin{cases}
            p>1,&\text{收敛},\\
            p \leq 1,&\text{发散}.
        \end{cases}&
        \text{广义P积分}\enspace& \int_{2}^{+\infty}\frac{1}{x \ln^p x}dx
        \begin{cases}
            p>1,&\text{收敛},\\
            p \leq 1,&\text{发散}.
        \end{cases}&\\
        \text{等比级数}\enspace& \sum_{n=1}^{\infty}aq^{n-1}
        \begin{cases}
            |q|<1,&\text{收敛},\\
            |q| \geq 1,&\text{发散}.
        \end{cases}&\\
    \end{align*} 

    \subsection{幂级数}

    \subsubsection{幂级数收敛域}

    \paragraph{阿贝尔定理} 当幂级数 \(\sum \limits_{n=1}^{\infty}a_nx^n\) 在点 \(x=x_1(x_1\neq 0)\) 处收敛时,对于满足 \(|x|<|x_1|\) 的一切 \(x\) ,幂级数绝对收敛;当幂级数 \(\sum \limits_{n=1}^{\infty}a_nx^n\) 在点 \(x=x_2(x_2\neq 0)\) 处发散时,对于满足 \(|x|>|x_2|\) 的一切 \(x\) ,幂级数发散。

    \paragraph{收敛域求法} 若 \(\lim \limits_{n \to \infty} \left| \frac{a_{n+1}}{a_{n}} \right| = \rho \) ,则 \(\sum \limits_{n=0}^{\infty}a_{n}x^{n}\) 的收敛半径 \(R\) 的表达式为
    \[R=\begin{cases}
        \frac{1}{\rho}, &\rho \neq 0,\\
        +\infty ,&\rho = 0,\\
        0 ,&\rho = +\infty.
    \end{cases} \]
    
    然后单独判断幂级数在 \(x=\pm R\) 处的敛散性,即可得到收敛域。

    \subsubsection{幂级数求和函数}
    
    \paragraph{两个特殊的幂级数}
    \begin{align*}
        \sum_{n=1}^{\infty}\frac{x^{n}}{n}&=-\ln (1-x) \enspace (-1 \leq  x < 1)&  \sum_{n=1}^{\infty} nx^{n-1}&=\frac{1}{(1-x)^{2}} \enspace (-1 < x < 1)&
    \end{align*}

    \subsubsection{函数展开为幂级数}
    这里指泰勒级数和麦克劳林级数,具体求法包括直接求泰勒公式法和利用已知幂级数展开式,通过变量代换、四则运算、逐项求导、逐项积分和待定系数法等方法得到函数的展开式。

    \subsection{傅里叶级数(p233)}

    \section{微积分几何应用}

    \subsection{一元函数微分学}

    \subsubsection{曲率半径}
    \paragraph{曲率公式} \[k=\frac{|y''|}{[1+(y')^2]^{3/2}}\]
    \paragraph{曲率半径公式} \[R=\frac{1}{k}\]

    \subsection{一元函数积分学}

    \subsubsection{平面曲线弧长}
    \paragraph{直角坐标方程} \(s=\int_{a}^{b} \sqrt{1+[y'(x)]^2}dx \). 
    \paragraph{参数方程} \(s=\int_{\alpha}^{\beta} \sqrt{\left(\frac{dx}{dt}\right)^2+\left(\frac{dy}{dt}\right)^2}dt \).
    \paragraph{极坐标方程} \(s=\int_{\alpha}^{\beta} \sqrt{[r(\theta)]^2+[r'(\theta)]^2}d\theta \).

    \subsubsection{旋转曲面的表面积}
    曲线 \(y=y(x)\) 绕 \(x\) 轴旋转一周所形成的旋转曲面表面积为 \(S=2 \pi \int_{a}^{b} \left| y(x) \right| \sqrt{1+[y'(x)^2]}dx \) 

    \subsection{多元函数的微分学}

    \subsubsection{旋转曲面}
    \paragraph{定义:} 曲线 \(\varGamma\) 绕一条定直线旋转一周所形成的曲面。
    
    曲线 \(\varGamma\) : 
    \(\begin{cases}
        F(x,y,z)=0,\\
        G(x,y,z)=0\\
    \end{cases} \)
    绕直线 \(L\): \(\frac{x-x_0}{m}=\frac{y-y_0}{n}=\frac{z-z_0}{p}\) 旋转一周所形成的曲面,求法如下:

    构建方程组,方程组包括三部分,分别是:曲线公式、垂直、等距

    \paragraph{垂直:} 在直线上取一点为 \(M_0\) ,在曲线上取一点为 \(M_1\) ,在 \(M_1\) 的纬圆(即 \(M_1\) 绕直线形成的圆)上取一点为 \(P\) ,则 \(M_1P\) 与直线垂直。
    \paragraph{等距:} \(M_0M_1\) 与 \(M_0P\) 等距。

    解方程组保留x,y,z即可得旋转曲面方程。
    
    \subsubsection{空间曲线和空间曲面的切线与法平面(p265)}
    
    \subsubsection{方向导数和梯度}
    \paragraph{方向导数} \(\frac{\partial \mathbf{u}}{\partial \mathbf{l}}= \mathbf{u}'_x(P_0)\cos \alpha + \mathbf{u}'_y(P_0)\cos \beta + \mathbf{u}'_z(P_0)\cos \gamma\)
    \paragraph{梯度} \(\mathbf{grad} u \left|_{P_0}\right. = \left(\mathbf{u}'_x(P_0), \mathbf{u}'_y(P_0), \mathbf{u}'_z(P_0)\right)\)
    \paragraph{注:} 梯度是一个向量,他的方向与是最大方向导数的方向,他的模是方向导数的最大值。
        
    \subsubsection{散度与旋度}
    \begin{align*}
        \text{散度} div \mathbf{A}& = \frac{\partial P}{\partial x} + \frac{\partial Q}{\partial y} + \frac{\partial R}{\partial z}&
        \text{旋度} \mathbf{rot} \mathbf{A}& = 
        \left|\begin{array}{cccc}
            \mathbf{i}& \mathbf{j}& \mathbf{k}\\
            \frac{\partial}{\partial x}& \frac{\partial}{\partial y}& \frac{\partial}{\partial z}\\
            P& Q& R
        \end{array}\right|&
    \end{align*}
    
    \subsection{多元函数积分学}
    
    \subsubsection{第一型曲线积分}
    \paragraph{投影根本规则} \(dL=\sqrt{(dx)^2+(dy)^2}\)
    
    \subsubsection{第一型曲面积分}
    \paragraph{投影根本规则} \(dS=\sqrt{(dxdy)^2+(dydz)^2+(dzdx)^2}\) 
    
    \subsubsection{形心公式(p282)、转动惯量(p283)、引力(p283)}

    \subsubsection{第二型曲线积分(平面)}

    \paragraph{格林公式} 设平面有界闭区域 \(D\) 由分段光滑的曲线 \(L\) 围成,函数 \(P(x, y)\) 和 \(Q(x, y)\) 在 \(D\) 上连续可偏导,\(L\) 取正向,则
    \[ \oint \limits_{L} Pdx+Qdy = \iint \limits_{D} \left( \frac{\partial Q}{\partial x} - \frac{\partial P}{\partial y} \right)d\sigma \]
    \paragraph{注:} \(L\) 取正向,即当一个人沿着 \(L\) 这个方向前进时,他的左手始终在 \(L\) 所围成的区域 \(D\) 内。
    
    \paragraph{积分与路径无关}
    \(\Rightarrow \begin{cases}
        Pdx+Qdy,& \text{是} f(x, y) \text{的全微分},\\
        \oint \limits_{L} Pdx+Qdx = 0,& \text{即} \frac{\partial Q}{\partial x} = \frac{\partial P}{\partial y}.
    \end{cases}\)

    \subsubsection{第二型曲面积分}
    \paragraph{基本公式}
    \begin{align*}
        \frac{dydz}{\cos \alpha}&=\frac{dzdx}{\cos \beta}=\frac{dxdy}{\cos \gamma}& \iint \limits_{\Sigma} R(x,y,z)dxdy&=\pm \iint \limits_{D_{xy}} R[x,y,z(x,y)]dxdy&
    \end{align*}

    \paragraph{高斯公式} 设空间有界闭区域 \(\Omega\) 由分片光滑的闭曲面 \(\Sigma \) 围成, \(P(x, y, z)\) , \(Q(x, y, z)\) , \(R(x, y, z)\) 在 \(\Omega\) 上具有一阶连续偏导数,\(\Sigma\) 取外侧,则
    \[\iint \limits_{\Sigma} Pdydz+Qdzdx+Rdxdy=\iiint \limits_{\Omega} \left( \frac{\partial P}{\partial x} + \frac{\partial Q}{\partial y} + \frac{\partial R}{\partial z} \right)dV^{}\]

    \subsubsection{第二型曲线积分(空间)}
    \paragraph{斯托克斯公式} 设 \(\Omega\) 为空间某区域, \(\Sigma \) 为 \(\Omega \) 内的分片光滑有向曲面片,\(\Gamma \) 为逐段光滑的 \(\Omega \) 的边界,它的方向与 \(\Sigma \) 的法向量成右手系,函数 \(P(x, y, z)\) , \(Q(x, y, z)\) , \(R(x, y, z)\) 在 \(\Omega\) 上具有一阶连续偏导数,则
    \[\oint_{\Gamma }Pdx+Qdy+Rdz = \iint \limits_{\Sigma }\left|\begin{array}{cccc}
        \mathbf{i}& \mathbf{j}& \mathbf{k}\\
        \frac{\partial}{\partial x}& \frac{\partial}{\partial y}& \frac{\partial}{\partial z}\\
        P& Q& R
    \end{array}\right|dS\]

\end{document}