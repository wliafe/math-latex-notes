\documentclass[a4paper,12pt,UTF8]{ctexart}
\usepackage{MyStyle}

\begin{document}
    \title{概率论知识点}
    \author{王泠风}
    \date{2024 年 10 月 07 日}
    \maketitle

    \section{随机事件和概率}

    \subsection{概率式}

    \subsubsection{证概率式等价}

    \paragraph{利用轮换对称性(2017年7题)}
    要证 \(P(A|B)>P(A|\overline{B})\) 的充要条件是 \(P(B|A)>P(B|\overline{A})\) ,发现他们两个是轮换对称的,因此一个成立另一个一定成立,互为充要条件。

    \section{随机变量的数字特征}

    \paragraph{遇到E(|X-Y|)X,Y为正态分布(李林2024年6一9)} 将 \(\left| X-Y \right| \) 看作一个整体,求其期望和方差,得出其概率密度,再通过概率密度求其绝对值期望。

\end{document}